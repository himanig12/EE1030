%ffalse
\let\negmedspace\undefined
\let\negthickspace\undefined
\documentclass[journal,12pt,twocolumn]{IEEEtran}
\usepackage{cite}
\usepackage{amsmath,amssymb,amsfonts,amsthm}
\usepackage{algorithmic}
\usepackage{graphicx}
\usepackage{textcomp}
\usepackage{xcolor}
\usepackage{txfonts}
\usepackage{listings}
\usepackage{enumitem}
\usepackage{mathtools}
\usepackage{gensymb}
\usepackage{comment}
\usepackage[breaklinks=true]{hyperref}
\usepackage{tkz-euclide} 
\usepackage{listings}
\usepackage{gvv}                                        
%\def\inputGnumericTable{}                                 
\usepackage[latin1]{inputenc}                                
\usepackage{color}                                            
\usepackage{array}                                            
\usepackage{longtable}                                       
\usepackage{calc}                                             
\usepackage{multirow}                                         
\usepackage{hhline}                                           
\usepackage{ifthen}                                           
\usepackage{lscape}
\usepackage{tabularx}
\usepackage{array}
\usepackage{float}


\newtheorem{theorem}{Theorem}[section]
\newtheorem{problem}{Problem}
\newtheorem{proposition}{Proposition}[section]
\newtheorem{lemma}{Lemma}[section]
\newtheorem{corollary}[theorem]{Corollary}
\newtheorem{example}{Example}[section]
\newtheorem{definition}[problem]{Definition}
\newcommand{\BEQA}{\begin{eqnarray}}
\newcommand{\EEQA}{\end{eqnarray}}
\newcommand{\define}{\stackrel{\triangle}{=}}
\theoremstyle{remark}
\newtheorem{rem}{Remark}

% Marks the beginning of the document
\begin{document}
\bibliographystyle{IEEEtran}
\vspace{3cm}

\title{18/A/E/36-49}
\author{AI24BTECH11011 - HIMANI GOURISHETTY}
\maketitle
\newpage
\bigskip

\renewcommand{\thefigure}{\theenumi}
\renewcommand{\thetable}{\theenumi}


\textbf{\textcolor{magenta}{36}}
\ Let $C_1$ and $C_2$ be the graphs of the functions $y=x^2$ and $y=2x$,$0\le x\le1$ respectively.Let $C_3$ be graph of a function y=f(x),$0\le x \le 1$, f(0)=0.For a point P on $C_1$,let the lines through P,parallel to the axes, meet $C_2$and $C_3$ at Q and R respectively(see figure).If for every position of P(on $C_1$), the areas of the shaded regions OPQ and ORP are equal, determine the function of f(x).
    \hfill{\textcolor{magenta}{(1998-8 Marks)}}

				        
					\textbf{\textcolor{magenta}{37}}
					   Integrate $\int_{0}^{\pi}\frac{e^{cosx}}{e^{cosx}+e^{-cosx}}$
					            \hfill{\textcolor{magenta}{(1999-2 Marks)}}\\
						             
							      \textbf{\textcolor{magenta}{38}}
							        Let f(x) be a continuos function given by \\
								     \[
								          f(x)=
									       \begin{cases}
									                2x,&|x|\le1\\
											         x^2+ax+b,&|x|>1
												      \end{cases}
												           \]
													    Find the area of the region in the third quadrant bounded by the curves x=-2$y^2$ and y=f(x) lying on the left of the line 8x+1=0.  
													        \hfill{\textcolor{magenta}{(1999-10marks)}}\\
														    
														    \textbf{\textcolor{magenta}{39}}
														    For $x > 0 $,$
														    let f(x)=\int_{e}^{x}\frac{lnt}{1+t}$
														    Find the function$ f(x) + f(\frac{1}{x})$
														    and show that $f(e)+f(\frac{1}{e})=
														    \frac{1}{2}$ .Here ,lnt=log t.
														      \hfill{\textcolor{magenta}{(2000-5Marks)}}
														        

															  \textbf{\textcolor{magenta}{40}}
															    Let $b\neq0$ and for j=0,1,2,...,n, $S_j$ be the area of the region bounded by the y-axis and the curve $xe^{ay}$ = sin by $\frac{jr}{b} \le y \le \frac{(j+1)\pi}{b}$ .Show that  $S_0,S_1,S_2,......,S_n$ are in geometric progression . Also , find their sum for a=-1and b=$pi$.\\.
															      \hfill{\textcolor{magenta}{(2001-5Marks)}}
															        
																  \textbf{\textcolor{magenta}{41}}
																    Find the area of the region bounded by the curves $y=x^2$ ,y=|2-x| and y=2, which lies to the right of the line.
																      \hfill{\textcolor{magenta}{(2002-5 Marks) }}

																        \textbf{\textcolor{magenta}{42}}
																	  If f is an even function then prove that 
																	    $\int_{0}^{\frac{\pi}{2}}f(cos2x)cox $ =$\sqrt{2}\int_{0}^{\frac{\pi}{4}}f(sin2x)cosx$\\.
																	      \hfill{\textcolor{magenta}{(2003-2Marks)}}
																	        
																		   \textbf{\textcolor{magenta}{43}}
																		      If $y(x)=\int_{\frac{\pi^2}{16}}^{x^2}\frac{cosxcos\sqrt{\theta}}{1+sin^2\sqrt{\theta}}$, then find $\frac{dy}{dx}$ at x=$pi$\\.
																		         \hfill{\textcolor{magenta}{(2004-2Marks)}}

																			    \textbf{\textcolor{magenta}{44}}
																			       Find the value of $\int_{\frac{-\pi}{3}}^{\frac{\pi}{3}}\frac{\pi+4x^3}{2-cos(|x|+\frac{\pi}{3})}$\\.
																			          \hfill{\textcolor{magenta}{(2004-4Marks)}}

																				     \textbf{\textcolor{magenta}{45}}
																				        Evaluate $\int_{0}^{\pi}e^{cosx}(2sin(\frac{1}{2}cosx)+3cos(\frac{1}{2}cosx))sinx$\\.
																					   \hfill{\textcolor{magenta}{(2005-2Marks)}}

																					      \textbf{\textcolor{magenta}{46}}
																					         
																						    Find the area bounded by the curves \\  
																						       $x^2=y,x^2=-y$ and $y^2=4x-3$
																						          \hfill{\textcolor{magenta}{(2005-4Marks)}}
																							     
																							        \textbf{\textcolor{magenta}{47}}
																								   
																								      f(x) is a differentiable function and g(x) is  double differentiable function such that $f(x)|\le1$ and f'(x)=g(x). if $f^2(0)+g^2(0)$=9. Prove that there exist some $c\in(-3,3)$ such that $g(c).g"(c)<0$.\\.
																								         \hfill{\textcolor{magenta}{(2005-6Marks)}}

																									    \textbf{\textcolor{magenta}{48}}
																									       \[
																									          \begin{bmatrix}
																										     4a^2 & 4a & 1\\
																										        4b^2 & 4b & 1\\
																											   4c^2 & 4c & 1\\
																											      \end{bmatrix}
																											         \begin{bmatrix}
																												        f(-1)\\
																													       f(1)\\
																													              f(2)
																														         \end{bmatrix}
																															    =
																															       \begin{bmatrix}
																															          3a^2+3a\\
																																     3b^2+3b\\
																																        3c^2+3c
																																	   \end{bmatrix}
																																	       \]
																																	          f(x) is a quadratic function and its maximum value occurs at a point V.A is a point of intersection of y=f(x) with x axis and point B is such that chord AB subtends a right angle at V . Find the area enclosed by f(x) and chord AB.
																																		    \hfill{\textcolor{magenta}{(2005-6Marks)}}

																																		      \textbf{\textcolor{magenta}{49}}
																																		        The value of 5050 $\frac{\int_{0}^{1}(1-x^50)^100}{\int{0}^{1} {(1-x^50)^101}}$
																																			  \hfill{\textcolor{magenta}{2006-6M}}

																																			     
																																			        
																																				  





																																				    \end{document}
