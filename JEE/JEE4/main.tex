%iffalse
\let\negmedspace\undefined
\let\negthickspace\undefined
\documentclass[journal,12pt,onecolumn]{IEEEtran}
\usepackage{cite}
\usepackage{amsmath,amssymb,amsfonts,amsthm}
\usepackage{algorithmic}
\usepackage{graphicx}
\usepackage{textcomp}
\usepackage{xcolor}
\usepackage{txfonts}
\usepackage{listings}
\usepackage{enumitem}
\usepackage{mathtools}
\usepackage{gensymb}
\usepackage{comment}
\usepackage[breaklinks=true]{hyperref}
\usepackage{tkz-euclide} 
\usepackage{listings}
\usepackage{gvv}                                        
%\def\inputGnumericTable{}                                 
\usepackage[latin1]{inputenc}                                
\usepackage{color}                                            
\usepackage{array}                                            
\usepackage{longtable}                                       
\usepackage{calc}                                             
\usepackage{multirow}                                         
\usepackage{hhline}                                           
\usepackage{ifthen}                                           
\usepackage{lscape}
\usepackage{tabularx}
\usepackage{array}
\usepackage{float}


\newtheorem{theorem}{Theorem}[section]
\newtheorem{problem}{Problem}
\newtheorem{proposition}{Proposition}[section]
\newtheorem{lemma}{Lemma}[section]
\newtheorem{corollary}[theorem]{Corollary}
\newtheorem{example}{Example}[section]
\newtheorem{definition}[problem]{Definition}
\newcommand{\BEQA}{\begin{eqnarray}}
\newcommand{\EEQA}{\end{eqnarray}}
\newcommand{\define}{\stackrel{\triangle}{=}}
\theoremstyle{remark}
\newtheorem{rem}{Remark}

% Marks the beginning of the document
\begin{document}
\bibliographystyle{IEEEtran}
\vspace{3cm}

\title{08-26-2021-shift-1-16-30}
\author{AI24BTECH11011 - Himani Gourishetty}
\maketitle
\bigskip

\renewcommand{\thefigure}{\theenumi}
\renewcommand{\thetable}{\theenumi}
\begin{enumerate}
        \item If $\vec{A}=\myvec{\frac{1}{\sqrt{5}}&\frac{2}{\sqrt{5}}\\\frac{-2}{\sqrt{5}}&\frac{1}{\sqrt{5}}},\vec{B}=\myvec{1&0\\i&1},i=\sqrt{-1},$ and $\vec{Q}=\vec{A^{T}}\vec{B}\vec{A}$, then the inverse of the matrix $\vec{A}\vec{Q^{2021}}\vec{A^{T}}$ is equal to:
        \begin{enumerate}
            \item $\myvec{\frac{1}{\sqrt{5}}&-2021\\2021&\frac{1}{\sqrt{5}}}$
            \item $\myvec{1&0\\-2021i&1}$
            \item $\myvec{1&0\\2021i&1}$
            \item $\myvec{1&-2021i\\0&1}$
        \end{enumerate}
        \item If the sum of an infinite GP $a, ar, ar^{2}, ar^{3},\cdots$ is 15 and the sum of the squares of its each term is 150, then the sum of $ar^{2}, ar^{4}, ar^{6}, \cdots$ is :
        \begin{enumerate}
            \item $\frac{5}{2}$
            \item $\frac{1}{2}$
            \item $\frac{25}{2}$
            \item $\frac{9}{2}$
        \end{enumerate}
        \item The value of $\lim_{n \to \infty}\frac{1}{n}\sum_{t=0}^{2n-1}\frac{n^{2}}{n^{2}+4r^{2}}$ is:
        \begin{enumerate}
            \item $\frac{1}{2}\tan^{-1}\brak{2}$
            \item $\frac{1}{2}\tan^{-1}\brak{4}$
            \item $\tan^{-1}\brak{4}$
            \item $\frac{1}{4}\tan^{-1}\brak{4}$
        \end{enumerate}
        \item Let $ABC$ be a triangle with $\vec{A}\brak{-3,1}$ and $\langle ACB=\theta,0<\theta<\frac{\pi}{2}$. If the equation of the median through $\vec{B}$ is $2x+y-3=0$ and the equation of the angle bisector of $\vec{C}$ is $7x-4y-1=0$, then $\tan \theta$ is equal to:
        \begin{enumerate}
            \item $\frac{1}{2}$
            \item $\frac{3}{4}$
            \item $\frac{4}{3}$
            \item 2
        \end{enumerate}
        \item If the truth value of the Boolean expression $\brak{\brak{p \lor q} \land \brak{q\rightarrow r} \land \brak{\neg r}}\rightarrow \brak{p \land q}$ is false, then truth values of the statements $p,q,r$ respectively can be:
        \begin{enumerate}
            \item T F T
            \item F F T
            \item T F F
            \item F T F
        \end{enumerate}
        \end{enumerate}
        \section{SECTION-B}
        \begin{enumerate}
		\item Let $z=\frac{1-i\sqrt{3}}{2},i=\sqrt{-1}$. Then the value of $21+\brak{z+\frac{1}{z}}^{3}+\brak{z^{2}+\frac{1}{z^{2}}}^{3}+\brak{z^{3}+\frac{1}{z^{3}}}^{3}+\cdots+\brak{z^{21}+\frac{1}{z^{21}}}^{3}$ is $\makebox[3cm][l]{\underline{\hspace{1cm}}}.$
        \item  The sum of all integral values of $k\brak{k\neq 0}$ for which the equation $\frac{2}{x-1}-\frac{1}{x-2}=\frac{2}{k}$ in $x$ has no real roots, is  $\makebox[3cm][l]{\underline{\hspace{1cm}}}$.
        \item Let the line $L$ be the projection of the line $\frac{x-1}{2}=\frac{y-3}{1}=\frac{z-4}{2}$ in the plane $x-2y-z=3$. If $d$ is the distance of the point $\brak{0, 0, 6}$ from $L$, then $d^{2}$ is equal to $\makebox[3cm][l]{\underline{\hspace{1cm}}}$.
       \item If $^{1}P_{1}+2\cdot^{2}P_{2}+3\cdot^{3}P_{3}+\cdots+15\cdot^{15}P_{15}=^{q}P_{r}=^{q}P_{r-s},0\leq s\leq1,$ the $^{q+s}C_{r-s}$ is equal to $\makebox[3cm][l]{\underline{\hspace{1cm}}}$.
       \item A wire of length 36 m is cut into two pieces, one of the pieces is bent to form a square and the other is bent to form a circle. If the sum of the areas of the two figures is minimum, and the circumference of the circle is $k (meter)$, then $\brak{\frac{4}{\pi}+1}k$ is equal to $\makebox[3cm][l]{\underline{\hspace{1cm}}}$.
       \item The area of the region $S=\{\brak{x,y}:3x^2 \leq 4y \leq 6x+24\}$ is $\makebox[3cm][l]{\underline{\hspace{1cm}}}$
       \item The locus of a point, which moves such that the sum of squares of its distances from the points $\brak{0, 0}, \brak{1, 0}, \brak{0, 1} \brak{1, 1}$ is 18 units, is a circle of diameter $d$. Then $d^2$ is equal to $\makebox[3cm][l]{\underline{\hspace{1cm}}}$.
       \item If $y=y\brak{x}$ is an implicit function of $x$ such that $\log_{e}\brak{x+y}=4xy,$ then $\frac{d^2y}{dx^2}$ at $x=0$ is equal to $\makebox[3cm][l]{\underline{\hspace{1cm}}}$.
       \item The number of three-digit even numbers, formed by the digits $0, 1, 3, 4, 6, 7$ if the repetition of digits is not allowed, is $\makebox[3cm][l]{\underline{\hspace{1cm}}}$.
       \item Let $a,b \in \mathbb{R}, b\neq 0$. Define a function 
       \[
       f\brak{x}=
       \begin{cases}
           a\sin\frac{\pi}{2}\brak{x-1},&\text{for }x\leq 0\\
           \frac{\tan 2x -\sin 2x}{bx^{3}},&\text{for }x>0
       \end{cases}
       \]
       If f is continous at $x=0$, then $10-ab$ is equal to  $\makebox[3cm][l]{\underline{\hspace{1cm}}}$.
\end{enumerate}
\end{document}}
