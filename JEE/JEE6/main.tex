%iffalse
\let\negmedspace\undefined
\let\negthickspace\undefined
\documentclass[journal,12pt,onecolumn]{IEEEtran}
\usepackage{cite}
\usepackage{amsmath,amssymb,amsfonts,amsthm}
\usepackage{algorithmic}
\usepackage{graphicx}
\usepackage{textcomp}
\usepackage{xcolor}
\usepackage{txfonts}
\usepackage{listings}
\usepackage{enumitem}
\usepackage{mathtools}
\usepackage{gensymb}
\usepackage{comment}
\usepackage[breaklinks=true]{hyperref}
\usepackage{tkz-euclide} 
\usepackage{listings}
\usepackage{gvv}                                        
%\def\inputGnumericTable{}                                 
\usepackage[latin1]{inputenc}                                
\usepackage{color}                                            
\usepackage{array}                                            
\usepackage{longtable}                                       
\usepackage{calc}                                             
\usepackage{multirow}                                         
\usepackage{hhline}                                           
\usepackage{ifthen}                                           
\usepackage{lscape}
\usepackage{tabularx}
\usepackage{array}
\usepackage{float}


\newtheorem{theorem}{Theorem}[section]
\newtheorem{problem}{Problem}
\newtheorem{proposition}{Proposition}[section]
\newtheorem{lemma}{Lemma}[section]
\newtheorem{corollary}[theorem]{Corollary}
\newtheorem{example}{Example}[section]
\newtheorem{definition}[problem]{Definition}
\newcommand{\BEQA}{\begin{eqnarray}}
\newcommand{\EEQA}{\end{eqnarray}}
\newcommand{\define}{\stackrel{\triangle}{=}}
\theoremstyle{remark}
\newtheorem{rem}{Remark}

% Marks the beginning of the document
\begin{document}
\bibliographystyle{IEEEtran}
\vspace{3cm}

\title{01-24-2023-shift-2-16-30}
\author{AI24BTECH11011 - Himani Gourishetty}
\maketitle
\bigskip

\renewcommand{\thefigure}{\theenumi}
\renewcommand{\thetable}{\theenumi}
\begin{enumerate}
\item Let $y = y\brak{x}$ be the solution of the differential
equation $\brak{x^2-3y^2}dx + 3xy dy = 0, y\brak{1} = 1$. Then
$6y^2\brak{e}$ is equal to
\begin{enumerate}
    \item $3e^2$
    \item $e^2$
    \item $2e^2$
    \item $\frac{3e^2}{2}$
\end{enumerate}
\item Let p and q be two statements . Then $\neg \brak{p \land \brak{p \Rightarrow \neg q}}$ is equivalent to
\begin{enumerate}
    \item $p \lor \brak{p \land \brak{\neg q}}$
    \item $p \lor \brak{\brak{\neg p} \land q}$
    \item $\brak{\neg p} \lor q$
    \item $p \lor \brak{p \land q}$
\end{enumerate}
\item The number of square matrices of order 5 with
entries from the set $\{0, 1\}$, such that the sum of all
the elements in each row is 1 and the sum of all the
elements in each column is also 1, is
\begin{enumerate}
    \item 225
    \item 120
    \item 150
    \item 125
\end{enumerate}
\item $\int_{\frac{3\sqrt{2}}{4}}^{\frac{3\sqrt{3}}{4}}\frac{48}{\sqrt{9-4x^2}}dx$ is equal to
\begin{enumerate}
    \item $\frac{\pi}{3}$
    \item $\frac{\pi}{2}$
    \item $\frac{\pi}{6}$
    \item $2\pi$
\end{enumerate}
\item Let $A$ be a $3 \times 3$ matrix such that $\abs{adj\brak{adj\brak{adjA}}}=12^4$.Then $\abs{A^{-1}adjA}$ is equal to 
\begin{enumerate}
    \item $2\sqrt{3}$
    \item $\sqrt{6}$
    \item 12
    \item 1
\end{enumerate}
\end{enumerate}
\section{SECTION-B}
\begin{enumerate}
    \item The urns $A, B$ and $C$ contain 4 red, 6 black; 5 red, 5 black and $\lambda$ red, 4 black balls respectively. One of the urns is selected at random and a ball is drawn. If the ball drawn is red and the probability that it is drawn from urn $C$ is 0.4 then the square of the length of the side of the largest equilateral triangle, inscribed in the parabola $y^2 = \lambda x$ with one vertex at the vertex of the parabola is
    \item If the area of the region bounded by the curves $y^2-2y=-x,x+y=0$ is $A$ , then $8A$ is equal to
    \item If $\frac{1^3+2^3+3^3+\cdots \text{upto n terms}}{1\cdot 3+2\cdot 5+3 \cdot 7+\cdots\text{upto n terms}}=\frac{9}{5}$, then the value of $n$ is
    \item If f be a differentiable function defined on $\lfloor 0,\frac{\pi}{2} \rfloor $ such that $f\brak{x}>$ and $f\brak{x}+\int_{0}^{x}f\brak{t}\sqrt{1-\brak{\log_ef\brak{t}}^2}dt=e,\forall x \in \lfloor 0,\frac{\pi}{2} \rfloor.$Then $\brak{6\log_ef\brak{\frac{\pi}{6}}}^2$ is equal to
    \item The minimum number of elements that must be added to the relation $R = \{\brak{a, b}, \brak{b, c}, \brak{b, d}\}$ on the set $\{a, b, c, d\}$ so that it is an equivalence relation, is $\makebox[3cm][l]{\underline{\hspace{1cm}}}$.
    \item Let $\vec{a}=\hat{i}+2\hat{j}+\lambda\hat{k},\vec{b}=3\hat{i}-5\hat{j}-\lambda\hat{k},\vec{a}\cdot\vec{c}=7,2\vec{b}\cdot\vec{c}+43=0,\vec{a}\times\vec{c}=\vec{b}\times\vec{c}.$ Then $\abs{\vec{a}\cdot\vec{b}}$ is equal to
    \item Let the sum of the coefficients of the first three terms in the expansion of $\brak{x-\frac{3}{x^2}}^n,x\neq 0,n \in N,$ be 376.Then the coefficient of $x^4$ is $\makebox[3cm][l]{\underline{\hspace{1cm}}}$.
    \item If the shortest distance between the lines $\frac{x
    +\sqrt{6}}{2}=\frac{y-\sqrt{6}}{3}=\frac{z-\sqrt{6}}{4}$ and $\frac{x-\lambda}{3}=\frac{y-2\sqrt{6}}{4}=\frac{z+2\sqrt{6}}{5}$ is 6, then square of sum of all possible values of $\lambda$ is 
    \item Let $S=\{\theta \in [0,2\pi):\tan\brak{\pi\cos\theta}+\tan\brak{\pi\sin\theta}=0\}$. Then $\sum_{\theta \in S}\sin^2\brak{\theta+\frac{\pi}{4}}$ is equal to
    \item The equations of the sides $AB,BC$ and $CA$ of a triangle $ABC $ are respectively: $2x=y=0,x+py=21a,\brak{a\neq 0}$ and $x-y=3$ respectively. Let $\vec{P}\brak{2,a}$ be the centroid of $\Delta ABC$. then $\brak{BC}^2$ is equal to
\end{enumerate}
\end{document}
