%iffalse
\let\negmedspace\undefined
\let\negthickspace\undefined
\documentclass[journal,12pt,onecolumn]{IEEEtran}
\usepackage{cite}
\usepackage{amsmath,amssymb,amsfonts,amsthm}
\usepackage{algorithmic}
\usepackage{graphicx}
\usepackage{textcomp}
\usepackage{xcolor}
\usepackage{txfonts}
\usepackage{listings}
\usepackage{enumitem}
\usepackage{mathtools}
\usepackage{gensymb}
\usepackage{comment}
\usepackage[breaklinks=true]{hyperref}
\usepackage{tkz-euclide} 
\usepackage{listings}
\usepackage{gvv}                                        
%\def\inputGnumericTable{}                                 
\usepackage[latin1]{inputenc}                                
\usepackage{color}                                            
\usepackage{array}                                            
\usepackage{longtable}                                       
\usepackage{calc}                                             
\usepackage{multirow}                                         
\usepackage{hhline}                                           
\usepackage{ifthen}                                           
\usepackage{lscape}
\usepackage{tabularx}
\usepackage{array}
\usepackage{float}


\newtheorem{theorem}{Theorem}[section]
\newtheorem{problem}{Problem}
\newtheorem{proposition}{Proposition}[section]
\newtheorem{lemma}{Lemma}[section]
\newtheorem{corollary}[theorem]{Corollary}
\newtheorem{example}{Example}[section]
\newtheorem{definition}[problem]{Definition}
\newcommand{\BEQA}{\begin{eqnarray}}
\newcommand{\EEQA}{\end{eqnarray}}
\newcommand{\define}{\stackrel{\triangle}{=}}
\theoremstyle{remark}
\newtheorem{rem}{Remark}

% Marks the beginning of the document
\begin{document}
\bibliographystyle{IEEEtran}
\vspace{3cm}

\title{01-08-2020-shift-2-1-15}
\author{AI24BTECH11011 - Himani Gourishetty}
\maketitle
\bigskip

\renewcommand{\thefigure}{\theenumi}
\renewcommand{\thetable}{\theenumi}
\begin{enumerate}
    \item Let $A$ and $B$ be two events such that the probability that exactly one of them occurs is $\frac{2}{5}$ and the probability that $A$ or $B$ occurs is $\frac{1}{2}$, then the probability of both of them occur together is
    \begin{enumerate}
	\item 0.10
        \item 0.20
        \item 0.01
        \item 0.02
    \end{enumerate}
    \item  Let $S$ be the set of all real roots of the equation, $3^x(3^x-1) + 2 = \abs{3^x-1} + \abs{3^x-2}$. Then $S$:
    \begin{enumerate}
        \item is a singleton.
        \item is an empty set.
        \item contains at least four elements.
        \item contains exactly two elements.
    \end{enumerate}
    \item The mean and variance of 20 observations are found to be 10 and 4, respectively. On rechecking, it was found that an observation 9 was incorrect and the correct observation was 11. Then the correct variance is:
    \begin{enumerate}
        \item 4.01
        \item 3.99
        \item 3.98
        \item 4.02
    \end{enumerate}
    \item Let $\vec{a}=\hat{i}-2\hat{j}+\hat{k},\vec{b}=\hat{i}-\hat{j}+\hat{k}$ be two vectors. If $\vec{c}$ is a vector such that $\vec{b}\times\vec{c} = \vec{b} \times \vec{a}$ and $\vec{a}.\vec{b}=0$ then $\vec{c}.\vec{b}$ is equal to:
    \begin{enumerate}
        \item $\frac{1}{2}$
        \item $\frac{-3}{2}$
        \item $\frac{-1}{2}$
        \item -1
    \end{enumerate}
    \item Let $f : \brak{1,3} \rightarrow \mathbb{R}$ be a function defined by $f\brak{x} = \frac{x\lfloor x \rfloor}{x^2+1}$ , where $\lfloor x \rfloor$ denotes the greatest integer $\leq x$. Then the range of f is:
    \begin{enumerate}
        \item $( \frac{2}{5} , \frac{3}{5} ] \cup ( \frac{3}{4} , \frac{4}{5} )$
        \item $( \frac{2}{5} , \frac{4}{5} ]$
        \item $( \frac{3}{5} , \frac{4}{5} )$
        \item $( \frac{2}{5} , \frac{1}{2} ) \cup ( \frac{3}{5} , \frac{4}{5} ]$
    \end{enumerate}
    \item If $\alpha$ and $\beta$ be the coefficients of $x^4$  and $x^2$ respectively in the expansion of $\brak{x+\sqrt{x^2-1}}^6 + \brak{x-\sqrt{x^2-1}}^6$, then:
    \begin{enumerate}
        \item $\alpha + \beta = -30 $
        \item $\alpha - \beta = -132$
       \item $\alpha + \beta = 60 $
        \item $\alpha - \beta = 60$
    \end{enumerate}
    \item  If a hyperbola passes through the point $\brak{10, 16}$ and it has vertices at $\brak{\pm 6, 0}$, then the equation of the normal at $\vec{P}$ is:
    \begin{enumerate}
        \item $3x+4y=94$
        \item $x+2y=42$
        \item $2x+5y=100$
        \item $x+3y=58$
    \end{enumerate}
    \item $\lim_{x \to 0} \frac{\int^{x}_0 t\sin\brak{10t}dt}{x}$ is equal to:
    \begin{enumerate}
        \item 0
        \item $\frac{1}{10}$
        \item $\frac{-1}{10}$
        \item $\frac{-1}{5}$
    \end{enumerate}
    \item If a line, $y= mx + c$ is a tangent to the circle, $\brak{x - 3}^2 + y^2 = 1$ and it is perpendicular to a line $L_1$, where $L_1$ is the tangent to the circle, $x^2 + y^2 = 1$ at the point $\brak{ \frac{1}{\sqrt{2}} , \frac{1}{\sqrt{2}}}$; then:
    \begin{enumerate}
        \item $c^2 + 7c + 6 = 0 $
        \item $c^2 - 6c + 7 = 0 $
        \item $c^2 - 7c + 6 = 0 $
        \item $c^2 + 6c + 7 = 0 $
    \end{enumerate}
    \item Let $\alpha=\frac{\brak{-1+i\sqrt{3}}}{2}$. If $\alpha=\brak{1+\alpha}\sum_{k=0}^{100}a^{2k}$ and $b=\sum_{k=0}^{100} a^{3k}$, then $a$ and $b$ are the roots of the quadratic equation:
    \begin{enumerate}
        \item $x^2+101x+100=0$
        \item $x^2+102x+101=0$
        \item $x^2-102x+101=0$
        \item $x^2-101x+100=0$
    \end{enumerate}
    \item  The mirror image of the point $\brak{1, 2, 3}$ in a plane is $\brak{\frac{-7}{3}, \frac{-4}{3} , \frac{-1}{3} }$. Which of the following points lies on this plane?
    \begin{enumerate}
        \item $\brak{1,-1,1}$
         \item $\brak{-1,-1,1}$
          \item $\brak{1,1,1}$
         \item $\brak{-1,-1,-1}$  
    \end{enumerate}
    \item The length of the perpendicular from the origin, on the normal to the curve,$x^2 + 2xy - 3y^2 = 0$ at the point $\brak{2, 2}$ is
    \begin{enumerate}
        \item 2
        \item $2\sqrt{2}$
        \item $4\sqrt{2}$
        \item $\sqrt{2}$
    \end{enumerate}
    \item Which of the following statements is a tautology?
    \begin{enumerate}
        \item $\neg\brak{p \land \neg q} \rightarrow \brak{p \lor q}$
        \item $\brak{\neg p \lor \neg q} \rightarrow \brak{p \land q}$
        \item $ p \land \brak{\neg q} \rightarrow \brak{p \land q}$
        \item $\neg\brak{p \lor \neg q} \rightarrow \brak{p \lor q}$
    \end{enumerate}
    \item If $I=\int_{1}^{2} \frac{dx}{\sqrt{2x^3-9x^2+12x+4}}$, then:
    \begin{enumerate}
        \item $\frac{1}{6}<I^2<\frac{1}{2}$
        \item $\frac{1}{8}<I^2<\frac{1}{4}$
        \item $\frac{1}{9}<I^2<\frac{1}{8}$
        \item $\frac{1}{16}<I^2<\frac{1}{9}$
    \end{enumerate}
    \item If $A = \myvec{2 & 2 \\ 9 & 4} $ and $I=\myvec{1 & 0 \\ 0 & 1}$, then $10 A^{-1}$ is equal to:
    \begin{enumerate}
        \item 6I-A
        \item A-6I
        \item 4I-A
         \item A-4I
    \end{enumerate}
    \end{enumerate}

\end{document} 
