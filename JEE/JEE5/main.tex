%iffalse
\let\negmedspace\undefined
\let\negthickspace\undefined
\documentclass[journal,12pt,onecolumn]{IEEEtran}
\usepackage{cite}
\usepackage{amsmath,amssymb,amsfonts,amsthm}
\usepackage{algorithmic}
\usepackage{graphicx}
\usepackage{textcomp}
\usepackage{xcolor}
\usepackage{txfonts}
\usepackage{listings}
\usepackage{enumitem}
\usepackage{mathtools}
\usepackage{gensymb}
\usepackage{comment}
\usepackage[breaklinks=true]{hyperref}
\usepackage{tkz-euclide} 
\usepackage{listings}
\usepackage{gvv}                                        
%\def\inputGnumericTable{}                                 
\usepackage[latin1]{inputenc}                                
\usepackage{color}                                            
\usepackage{array}                                            
\usepackage{longtable}                                       
\usepackage{calc}                                             
\usepackage{multirow}                                         
\usepackage{hhline}                                           
\usepackage{ifthen}                                           
\usepackage{lscape}
\usepackage{tabularx}
\usepackage{array}
\usepackage{float}


\newtheorem{theorem}{Theorem}[section]
\newtheorem{problem}{Problem}
\newtheorem{proposition}{Proposition}[section]
\newtheorem{lemma}{Lemma}[section]
\newtheorem{corollary}[theorem]{Corollary}
\newtheorem{example}{Example}[section]
\newtheorem{definition}[problem]{Definition}
\newcommand{\BEQA}{\begin{eqnarray}}
\newcommand{\EEQA}{\end{eqnarray}}
\newcommand{\define}{\stackrel{\triangle}{=}}
\theoremstyle{remark}
\newtheorem{rem}{Remark}

% Marks the beginning of the document
\begin{document}
\bibliographystyle{IEEEtran}
\vspace{3cm}

\title{07-26-2022-shift-2-16-30}
\author{AI24BTECH11011 - Himani Gourishetty}
\maketitle
\bigskip

\renewcommand{\thefigure}{\theenumi}
\renewcommand{\thetable}{\theenumi}
\begin{enumerate}
    \item If $0<x<\frac{1}{\sqrt{2}}$ and $\frac{\sin^{-1}x}{\alpha}=\frac{\cos^{-1}x}{\beta}$, then a value of $\sin\brak{\frac{2\pi\alpha}{\alpha+\beta}}$ is 
	    \hfill{(July 2022)}
	\begin{enumerate}
        \item $4\sqrt{\brak{1-x^2}}\brak{1-2x^2}$
        \item $4x\sqrt{\brak{1-x^2}}\brak{1-2x^2}$
        \item $2x\sqrt{\brak{1-x^2}}\brak{1-4x^2}$
        \item $4\sqrt{\brak{1-x^2}}\brak{1-4x^2}$
    \end{enumerate}
    \item Negation of the Boolean expression $p \Leftrightarrow \brak{q \Rightarrow p}$ is
    \hfill{(July 2022)} 
	\begin{enumerate}
        \item $\brak{\neg p} \land q$
        \item $p \land \brak{\neg q}$
        \item $\brak{\neg p} \lor \brak{\neg q}$
        \item $\brak{\neg p}\land \brak{\neg q}$
    \end{enumerate}
    \item Let $X$ be a binomially distributed random variable with mean 4 and variance $\frac{4}{3}$. Then $54 P \brak{X \leq 2}$ is equal to
    \hfill{(July 2022)} 
	\begin{enumerate}
        \item $\frac{73}{27}$
        \item $\frac{146}{27}$
        \item $\frac{146}{81}$
        \item $\frac{126}{81}$
    \end{enumerate}
    \item The integral $\int\frac{\brak{1-\frac{1}{\sqrt{3}}}\brak{\cos x - \sin x}}{\brak{1+\frac{2}{\sqrt{3}}\sin 2x}} dx$ is equal to 
   \hfill{(July 2022)} 
	\begin{enumerate}
        \item $\frac{1}{2}\log_e\abs{\frac{\tan\brak{\frac{x}{2}+\frac{\pi}{12}}}{\frac{x}{2}+\frac{\pi}{6}}} + C $
        \item $\frac{1}{2}\log_e\abs{\frac{\tan\brak{\frac{x}{2}+\frac{\pi}{6}}}{\frac{x}{2}+\frac{\pi}{3}}} + C $
        \item $\log_e\abs{\frac{\tan\brak{\frac{x}{2}+\frac{\pi}{6}}}{\frac{x}{2}+\frac{\pi}{12}}} + C $
        \item $\frac{1}{2}\log_e\abs{\frac{\tan\brak{\frac{x}{2}-\frac{\pi}{12}}}{\frac{x}{2}-\frac{\pi}{6}}} + C $
    \end{enumerate}
    \item The area bounded by the curves $y=\abs{x^2-1}$ and $y=1$ is 
   \hfill{(July 2022)} 
\begin{enumerate}
        \item $\frac{2}{3}\brak{\sqrt{2}+1}$
        \item $\frac{4}{3}\brak{\sqrt{2}-1}$
        \item $2\brak{\sqrt{2}-1}$
        \item $\frac{8}{3}\brak{\sqrt{2}-1}$
    \end{enumerate}
    \end{enumerate}
    \section{SECTION-B}
    \begin{enumerate}
        \item Let $A=\{1,2,3,4,5,6,7\}$ and $B=\{3,6,7,9\}$. Then the number of elements in the set $\{C\subseteq A: C \cap B \neq \phi\}$ is $\makebox[3cm][l]{\underline{\hspace{1cm}}}$.
 \hfill{(July 2022)} 
 \item The largest value of $a$, for which the perpendicular distance of the plane containing the lines $r=\brak{\hat{i}+\hat{j}}+\lambda\brak{\hat{i}+a\hat{j}-\hat{k}}$ and $r=\brak{\hat{i}+\hat{j}}+\mu\brak{-\hat{i}+\hat{j}-a\hat{k}}$ from the point $\brak{2,1,4}$ is $\sqrt{3}$, is $\makebox[3cm][l]{\underline{\hspace{1cm}}}$.  
 \hfill{(July 2022)} 
 \item Numbers are to be formed between 1000 and 3000, which are divisible by 4, using the digits 1,2,3,4,5 and 6 without repetition of digits. Then the total number of such numbers is $\makebox[3cm][l]{\underline{\hspace{1cm}}}$
	\hfill{(July 2022)} 
 \item If $\sum_{k=1}^{10}\frac{k}{k^4+k^2+1}=\frac{m}{n}$, where $m$ and $n$ are co-prime, then $m+n$ is equal to
 \hfill{(July 2022)} 
 \item If the sum of solutions of the system of equations $2sin^{2}\theta-\cos\theta=0$ and $2\cos^{2}\theta+3\sin \theta=0$ in the interval $\lfloor 0,2\pi \rfloor$ is $k\pi$, then $k$ is equal to $\makebox[3cm][l]{\underline{\hspace{1cm}}}$.
 \hfill{(July 2022)} 
 \item The mean and standard deviation of 40 observations are 30 and 5 respectively. It was noticed that two of these observations 12 and 10 were wrongly recorded. If $\sigma$ is the standard deviation of the data after omitting the two wrong observations from the data, then $38\sigma^{2}$ is equal to $\makebox[3cm][l]{\underline{\hspace{1cm}}}$.
 \hfill{(July 2022)} 
 \item The plane passing through the line : $L:lx-y+3\brak{1-l}z=1,x+2y-z=2$ and perpendicular to the plane $3x+2y+z=6$ is $3x-8y+7z=4$. If $\theta$ is the acute angle between the line $L$ and the y-axis, then $415\cos^{2}\theta$ is equal to $\makebox[3cm][l]{\underline{\hspace{1cm}}}$.
 \hfill{(July 2022)} 
 \item Suppose $y=y\brak{x}$ be the solution curve to the differential equation $\frac{dy}{dx}-y=2-e^{-x}$ such that $\lim_{x \to \infty} y\brak{x}$ is finite. If $a$ and $b$ are respectively the x and y-intercept of the tangent to the curve at $x=0$, then the value of $a-4b$ is equal to $\makebox[3cm][l]{\underline{\hspace{1cm}}}$.
 \hfill{(July 2022)} 
 \item Different A.P.'s are constructed with the first term 100, the last term 199, And integral common differences. The sum of the common differences of all such, A.P's having at least 3 terms and at most 33 terms is.
 \hfill{(July 2022)} 
 \item The number of matrices $\vec{A}=\myvec{a&b\\c&d}$, where $a,b,c,d \in\{-1,0,1,2,3,4,\cdots,10\}$, such that $\vec{A}=\vec{A^{-1}}$, is $\makebox[3cm][l]{\underline{\hspace{1cm}}}$.
 \hfill{(July 2022)} 
    \end{enumerate}
\end{document}h
