%iffalse
\let\negmedspace\undefined
\let\negthickspace\undefined
\documentclass[journal,12pt,onecolumn]{IEEEtran}
\usepackage{cite}
\usepackage{amsmath,amssymb,amsfonts,amsthm}
\usepackage{algorithmic}
\usepackage{graphicx}
\usepackage{textcomp}
\usepackage{xcolor}
\usepackage{txfonts}
\usepackage{listings}
\usepackage{enumitem}
\usepackage{mathtools}
\usepackage{gensymb}
\usepackage{comment}
\usepackage[breaklinks=true]{hyperref}
\usepackage{tkz-euclide} 
\usepackage{listings}
\usepackage{gvv}       
\usepackage{circuitikz}
%\def\inputGnumericTable{}                                 
\usepackage[latin1]{inputenc}                                
\usepackage{color}                                            
\usepackage{array}                                            
\usepackage{longtable}                                       
\usepackage{calc}                                             
\usepackage{multirow}                                         
\usepackage{hhline}                                           
\usepackage{ifthen}                                           
\usepackage{lscape}
\usepackage{tabularx}
\usepackage{array}
\usepackage{float}


\newtheorem{theorem}{Theorem}[section]
\newtheorem{problem}{Problem}
\newtheorem{proposition}{Proposition}[section]
\newtheorem{lemma}{Lemma}[section]
\newtheorem{corollary}[theorem]{Corollary}
\newtheorem{example}{Example}[section]
\newtheorem{definition}[problem]{Definition}
\newcommand{\BEQA}{\begin{eqnarray}}
\newcommand{\EEQA}{\end{eqnarray}}
\newcommand{\define}{\stackrel{\triangle}{=}}
\theoremstyle{remark}
\newtheorem{rem}{Remark}

% Marks the beginning of the document
\begin{document}
\bibliographystyle{IEEEtran}
\vspace{3cm}

\title{2022-PH-27-39}
\author{AI24BTECH11011 - Himani Gourishetty}
\maketitle
\bigskip

\renewcommand{\thefigure}{\theenumi}
\renewcommand{\thetable}{\theenumi}
\begin{enumerate}

\item The ordinary differential equation
$$\brak{1 - x^2}y'' - xy' + 9y = 0$$
has a regular singularity at
\begin{enumerate}
    \item -1
    \item 0
    \item +1
    \item no finite value of x
\end{enumerate}

\item For a bipolar junction transistor, which of the following statements are true?
\begin{enumerate}
    \item Doping concentration of emitter region is more than that in collector and base region
    \item Only electrons participate in current conduction
    \item The current gain $\beta$ depends on temperature
    \item Collector current is less than the emitter current
\end{enumerate}

\item Potassium metal has electron concentration of $1.4 \times 10^{28}$  $m^{-3}$ and the corresponding density of states at Fermi level is $6.2 \times 10^{46}$ $Joule^{-1} m^{-3}$. If the Pauli paramagnetic susceptibility of Potassium is $n\times 10^{-k}$ in standard scientific form, then the value of k $\brak{\text{an integer}}$ is $\makebox[2cm][l]{\underline{\hspace{2cm}}}$

$\brak{\text{Magnetic moment of electron is } 9.3 \times 10^{-24} Joule T^{-1} ; \text{ permeability of free space is } 4\pi \times 10^{-7} TmA^{-1}}$

\item A power supply has internal resistance $R_S$ and open load voltage $V_S= 5V$. When a load resistance $R_L$ is connected the power supply, a voltage drop of $V_L = 4V$ is measured across the load. The value of $\frac{R_L}{R_S}$ is $\makebox[2cm][l]{\underline{\hspace{2cm}}}$
$\brak{\text{Round off to the nearest integer}}$

\item Electric field is measured along the axis of a uniformly charged disc of radius 25 cm. At a distance d from the centre, the field differs by $10\%$ from that of a infinite plane having same charge density. The value of d is $\makebox[1cm][l]{\underline{\hspace{1cm}}}$ cm.
$\brak{\text{Round off to one decimal place}}$

\item In a solid, a Raman line observed at $300 cm^{-1}$ has intensity of Stokes line four times that of the anti-Stokes line. The temperature of the sample is $\makebox[1cm][l]{\underline{\hspace{1cm}}}$K.
$\brak{\text{Round of to nearest integer}}$
$\brak{1cm^{-1} \equiv 1.44K}$

\item An electromagnetic pulse has pulse width of $10^{-3} s$. The uncertainty in the momentum of the correspoding photo is of the order of $10^{-N} kg m s^{-1}$, where N is an integer. The value of N is $\makebox[3cm][l]{\underline{\hspace{3cm}}}$ 
$\brak{\text{speed of light}= 3 \times 10^8 m s^{-1}, h= 6.6 \times 10^-34 Js}$

\item The wavefunction of a particle in a one-dimensional infinite well of size 2a at a certain time is $\varphi \brak{x} = \frac{1}{\sqrt{6a}}[\sqrt{2}sin\brak{\frac{\pi x}{a}} + \sqrt{3}cos\brak{\frac{\pi x}{2a}} + cos\brak{\frac{3\pi x}{2a}}]$. Probability of finding the particle in $n=2$ state at that time is $\makebox[1cm][l]{\underline{\hspace{1cm}}}$$\%$ 
$\brak{\text{Round off to the nearest integer}}$

\item A spectrometer is used to detect plasma oscillations in a sample. The spectrometer can work in the range of $3 \times 10^{12} rad s^{-1}$ to $30 \times 10^{12} rad s^{-1}$. The minimum carrier concentration that can be detected by using this spectrometer is $n \times 10^{21} m^{-3}$. The value of n is $\makebox[2cm][l]{\underline{\hspace{2cm}}}$
$\brak{\text{Round off to two decimal places}}$
$\brak{\text{Charge of an electron }= -1.6 \times 10^{-1}C, \text{mass of an electron }= 9.1 \times 10^{-31}kg \text{ and } \epsilon_0 = 8.85 \times 10^{-12} C^2 N^{-1} m^{-2}}$

\item Consider a non-interacting gas of spin 1 particles, each with magnetic moment $\mu$, placed in a weak magnetic field B, such that $\frac{\mu B}{k_B T} << 1$. The average magnetic moment of a particle is 
\begin{enumerate}
    \item $\frac{2\mu}{3} \brak{\frac{\mu B}{k_B T}}$
    \item $\frac{\mu}{2} \brak{\frac{\mu B}{k_B T}}$
    \item $\frac{\mu}{3} \brak{\frac{\mu B}{k_B T}}$
    \item $\frac{3\mu}{4} \brak{\frac{\mu B}{k_B T}}$
\end{enumerate}

\item Water at 300 K can be brought to 320 K using one of the following processes.\\
Process 1: Water is brought in equilibrium with a reservoir at 320 K directly.\\
Process 2: Water is first brought in equilibrium with a reservoir at 310 K and then with the reservoir at 320 K.\\
Process 3: Water is first brought in equilibrium with a reservoir at 350 K and then with the reservoir at 320 K.\\
The corresponding changes in the entropy of the universe for these processes are $\Delta S_1$, $\Delta S_2$, $\Delta S_3$ respectively. Then 
\begin{enumerate}
    \item $\Delta S_2 > \Delta S_1 > \Delta S_3$
    \item $\Delta S_3 > \Delta S_1 > \Delta S_2$
    \item $\Delta S_3 > \Delta S_2 > \Delta S_1$
    \item $\Delta S_1 > \Delta S_2 > \Delta S_3$
\end{enumerate}

\item A student sets up Young's double slit experiment with electrons of momentum p incident normally on the slits with width w separated by distance d. In order to observe interference fringes on a screen at a distance D from the slits, which of the following conditions should be satisfied?
\begin{enumerate}
    \item $\frac{h}{p} > \frac{Dw}{d}$
    \item $\frac{h}{p} > \frac{dw}{D}$
    \item $\frac{h}{p} > \frac{d^2}{D}$
    \item $\frac{h}{p} > \frac{d^2}{\sqrt{Dw}}$
\end{enumerate}

\item Consider a particle in three different boxes of width L. The potential inside the boxes vary as shown in figures \brak{i}, \brak{ii} and \brak{iii} with $V_0 << \frac{h^2 \pi ^2}{2m L^2}$. The corresponding ground-state energies of the particle are $E_1$, $E_2$ and $E_3$, respectively. Then
\begin{figure}[!ht]
\centering
\resizebox{0.5\textwidth}{!}{%

\begin{circuitikz}
\tikzstyle{every node}=[font=\normalsize]
\draw [->, >=Stealth, dashed] (2.5,8.75) -- (2.5,15.75);
\draw [->, >=Stealth, dashed] (2.5,9) -- (20,9);
\draw [dashed] (2.5,10.5) -- (19.75,10.5);
\draw [->, >=Stealth] (3.5,10.5) -- (3.5,14.5);
\draw [short] (3.5,10.5) -- (5,10.5);
\draw [short] (5,10.5) -- (5,9);
\draw [short] (5,9) -- (7.5,9);
\draw [->, >=Stealth] (7.5,9) -- (7.5,14.5);
\draw [->, >=Stealth] (9.75,9) -- (9.75,14.5);
\draw [short] (9.75,9) -- (11,9);
\draw [short] (11,9) -- (11,10.5);
\draw [short] (11,10.5) -- (12.5,10.5);
\draw [short] (12.5,10.5) -- (12.5,9);
\draw [short] (12.5,9) -- (13.75,9);
\draw [->, >=Stealth] (14,9) -- (14,14.5);
\draw [->, >=Stealth] (15.5,10.5) -- (15.5,14.5);
\draw [short] (15.5,10.5) -- (16,10.5);
\draw [short] (16,10.5) -- (16,9);
\draw [short] (16,9) -- (18.5,9);
\draw [short] (18.5,9) -- (18.5,10.5);
\draw [short] (18.5,10.5) -- (19,10.5);
\draw [->, >=Stealth] (19,10.5) -- (19,14.5);
\node [font=\normalsize] at (2,10.25) {V};
\node [font=\scriptsize] at (2.25,10.25) {0};
\node [font=\normalsize] at (5.5,13.5) {(i)};
\node [font=\normalsize] at (12,13.5) {(ii)};
\node [font=\normalsize] at (17,13.5) {(iii)};
\node [font=\normalsize] at (4.25,11) {L/3};
\node [font=\normalsize] at (10.25,11) {L/3};
\node [font=\normalsize] at (11.75,11) {L/3};
\node [font=\normalsize] at (15.75,11) {L/6};
\node [font=\normalsize] at (17.25,9.5) {2L/3};
\draw [<->, >=Stealth] (3.75,10.75) -- (4.75,10.75);
\draw [<->, >=Stealth] (9.75,10.75) -- (10.75,10.75);
\draw [<->, >=Stealth] (11,10.75) -- (12.5,10.75);
\draw [<->, >=Stealth] (15.5,10.75) -- (16,10.75);
\draw [<->, >=Stealth] (16.25,9.25) -- (18.25,9.25);
\node [font=\normalsize] at (2.25,9) {0};
\end{circuitikz}

}%
\end{figure}
\begin{enumerate}
    \item $E_2 > E_1 > E_3$
    \item $E_3 > E_1 > E_2$
    \item $E_2 > E_3 > E_1$
    \item $E_3 > E_2 > E_1$
\end{enumerate}



\end{enumerate}
\end{document}
