%iffalse
\let\negmedspace\undefined
\let\negthickspace\undefined
\documentclass[journal,12pt,onecolumn]{IEEEtran}
\usepackage{cite}
\usepackage{amsmath,amssymb,amsfonts,amsthm}
\usepackage{algorithmic}
\usepackage{graphicx}
\usepackage{textcomp}
\usepackage{xcolor}
\usepackage{txfonts}
\usepackage{listings}
\usepackage{enumitem}
\usepackage{mathtools}
\usepackage{gensymb}
\usepackage{comment}
\usepackage[breaklinks=true]{hyperref}
\usepackage{tkz-euclide} 
\usepackage{listings}
\usepackage{gvv}       
\usepackage{circuitikz}
%\def\inputGnumericTable{}                                 
\usepackage[latin1]{inputenc}                                
\usepackage{color}                                            
\usepackage{array}                                            
\usepackage{longtable}                                       
\usepackage{calc}                                             
\usepackage{multirow}                                         
\usepackage{hhline}                                           
\usepackage{ifthen}                                           
\usepackage{lscape}
\usepackage{tabularx}
\usepackage{array}
\usepackage{float}


\newtheorem{theorem}{Theorem}[section]
\newtheorem{problem}{Problem}
\newtheorem{proposition}{Proposition}[section]
\newtheorem{lemma}{Lemma}[section]
\newtheorem{corollary}[theorem]{Corollary}
\newtheorem{example}{Example}[section]
\newtheorem{definition}[problem]{Definition}
\newcommand{\BEQA}{\begin{eqnarray}}
\newcommand{\EEQA}{\end{eqnarray}}
\newcommand{\define}{\stackrel{\triangle}{=}}
\theoremstyle{remark}
\newtheorem{rem}{Remark}

% Marks the beginning of the document
\begin{document}
\bibliographystyle{IEEEtran}
\vspace{3cm}

\title{2015-CE-53-65}
\author{AI24BTECH11011 - Himani Gourishetty}
\maketitle
\bigskip

\renewcommand{\thefigure}{\theenumi}
\renewcommand{\thetable}{\theenumi}
\begin{enumerate}


\item Two reservoirs are connected through a 930 m long, 0.3 m diameter pipe, which has a gate valve. The pipe entrance is sharp $\brak{\text{loss coefficient} = 0.5}$ and the value is half-open $\brak{\text{loss coefficient} = 0.5}$. The head difference between the two reservoirs is 20 m. Assume the friction factor for the pipe as 0.03 and $g = 10 \frac{m}{s^2}$. The discharge in the pipe accounting for all minor and major losses is $\makebox[3cm][l]{\underline{\hspace{3cm}}}$ $\frac{m^3}{s}$

\item A hydraulic jump is formed in a 2 m wide rectangular channel which is horizontal and frictionless. The post-jump depth and velocity are 0.8 m and 1 $\frac{m}{s}$, respectively. The pre-jump velocity is $\makebox[3cm][l]{\underline{\hspace{3cm}}}$ $\frac{m}{s}$. \brak{\text{use g} = 10 \frac{m}{s^2}}

\item A short reach of a 2 m wide rectangular open channel has its bed level rising in the direction flow at a slope 1 in 10000. it carries a discharge of 4 $\frac{m^3}{s}$ and its Manning's roughness coefficient is 0.01. The flow in this reach is gradually varying. At a certain section in this reach, the depth of flow was measured as 0.5m. The rate of change of the water depth and distance, $\frac{dy}{dx}$, at this section is $\makebox[3cm][l]{\underline{\hspace{1cm}}}$ $\brak{\text{use g} = 10 \frac{m}{s^2}}$

\item The drag force, $F_D$, on a sphere kept in a uniform flow field depends on the diameter of the sphere, $D$, flow velocity, $V$, fluid density, $\rho$,and dynamic viscosity, $\mu$. Which of the following options represents the non-dimensional parameters which could be used to analyze this problem?
\begin{enumerate}
    \item $\frac{F_D}{VD}$ and $\frac{\mu}{{\rho}VD}$
    \item $\frac{F_D}{{\rho}V{R^2}}$ and $\frac{{\rho}VD}{\mu}$
    \item $\frac{F_D}{{\rho}{V^2}[D^2]}$ and $\frac{{\rho}VD}{\mu}$
    \item $\frac{F_D}{{\rho}{V^3}{D^3}}$ and $\frac{\mu}{{\rho}VD}$

\end{enumerate}

\item In in catchment, there are four rain-gauge stations, P, Q, R, and S. Normal annual precipitation values at these stations are 780 mm, 850 mm,920 mm, and 980 mm, respectively. In the year 2013, stations Q, R, and S, were operative but P was not. Using the normal raio method, the precipitation at station P for the year 2013 has been estimated as 860 mm. If the observed precipitation at stations Q and R for the year 2013 were 930 mm and 1010 mm, respectively; what was the observed precipitation \brak{\text{in mm}} at station S for that year?

\item The 4-hr unit hydrograph for a catchment is given in the table below. What would be the maximum ordinate of the S-curve $\brak{\frac{m^3}{s}}$ derived from this hydrograph?
\begin{table}[h!]    
\centering
\begin{circuitikz}
\tikzstyle{every node}=[font=\small]
\draw (6.5,11.5) to[short] (8.25,11.5);
\draw (8.25,11.5) to[C] (11,11.5);
\draw (11,11.5) to[R] (12.75,11.5);
\draw [short] (12.75,11.5) -- (15.75,11.5);
\draw (15.75,8.25) to[american current source] (15.75,11.5);
\draw [short] (15.75,8.25) -- (6.5,8.25);
\draw (8.25,11.5) to[C] (8.25,8.25);
\draw (6.5,11.5) to[C] (6.5,8.25);
\node at (8.25,11.5) [circ] {};
\node at (8.25,8.25) [circ] {};
\draw (14,11.5) to[R] (14,8.25);
\node at (14,11.5) [circ] {};
\node at (14,8.25) [circ] {};
\node [font=\small] at (9.75,12) {1F};
\node [font=\small] at (8.75,9.75) {1F};
\node [font=\small] at (6,9.75) {1F};
\node [font=\small] at (11.75,12) {3$\Omega$};
\node [font=\small] at (13.5,9.75) {3$\Omega$};
\node [font=\small] at (16.5,9.75) {3A};
\end{circuitikz}


\end{table}
\item The concentration of the Sulfur Dioxide $\brak{SO_2}$ in ambient atmosphere was measured at $ 30 \frac{\mu g}{m^3}$. Under the same conditions, the above $SO_2$ concentration is expressed in ppm is  $\makebox[3cm][l]{\underline{\hspace{3cm}}}$\\
Given : $\frac{P}{\brak{RT}}=41.6\frac{mol}{m^3}$,where, $P$=Pressure, $T$=Temperature, $R$=universal gas constant,  Molecular weight of $SO_2=64$
\item Consider a primary sedimentary tank $\brak{PST}$ in a water treatment plant with  Sulfur Overflow Rate$\brak{SOR}$ of $40m^3/m^2/d.$ The diameter of the spherical particle which will have 90 percent theoretical removal efficiency in this tank is $\makebox[3cm][l]{\underline{\hspace{3cm}}}$ $\mu m$. Assume the setting velocity of the particles in water is describes by Stroke's Law.\\
Given : Density of the water $= 1000 \frac{kg}{m^3}$; Density of the particle $=2650 \frac{kg}{m^3}$;$g=9.81 \frac{m}{s^2}$; Kinematic velocity of the water particle$\brak{v}=1.10 \times 10^{-6} \frac{m^2}{s}$
\item The acceleration-time relationship for a vehicle is subjected to non-uniform acceleration is, $$\frac{dv}{dt}=\brak{\alpha - \beta v_0} e^{-\beta t}$$ where, v is the speed in $\frac{m}{s}$, t is the time in s, $\alpha$ and $\beta$ are parameters, and $v_0$ is the initial speed in $\frac{m}{s}$. If the accelerating behavior of the vehicle, whose drivers intends to overtake a slow moving vehicle ahead, is described as, $$\frac{dv}{dt} = \brak{\alpha - {\beta}v}$$

considering $\alpha = 2 \frac{m}{s^2}$, $\beta = 0.05 s^{-1}$ and $\frac{dv}{dt} = 1.3 \frac{m}{s^2}$ at $t = 3 s$, the distance $\brak{\text{in m}}$ travelled by the vehicle in 35 s is $\makebox[3cm][l]{\underline{\hspace{3cm}}}$.

\item On a circular curve, the rate of superelevation is e. While negotiating the curve a vehicle comes to a stop. It is seen that the stopped vehicle does not slide inwards $\brak{\text{in the radial direction}}$. The coefficient of side function is f. Which of the following is true:
\begin{enumerate}
    \item $e \leq f$
    \item $f < e < 2f$
    \item $e \geq ef$
    \item none of these
\end{enumerate}
\item A sign is required to be put asking the drivers to slow down to 30 $\frac{km}{h}$
before entering Zone Y $\brak{\text{see figure}}$. On this road, vehicles require 174 m to slow down to 30 $\frac{km}{h}$ $\brak{\text{the distance of 174 m includes the distance travelled during the perception-reaction time of drivers}}$. The sign can be read by $\frac{6}{6}$ vision drivers from a distance of 48m. The sign is placed at a distance of x m from the start of Zone Y so that even a $\frac{6}{9}$ vision driver can slown to 30 $\frac{km}{h}$ before entering the zone. The minimum value of x is $\makebox[3cm][l]{\underline{\hspace{3cm}}} $m.\\
\begin{figure}[!ht]
\centering
\resizebox{0.4\textwidth}{!}{%

\begin{circuitikz}
\tikzstyle{every node}=[font=\normalsize]
\draw [->, >=Stealth, dashed] (2.5,8.75) -- (2.5,15.75);
\draw [->, >=Stealth, dashed] (2.5,9) -- (20,9);
\draw [dashed] (2.5,10.5) -- (19.75,10.5);
\draw [->, >=Stealth] (3.5,10.5) -- (3.5,14.5);
\draw [short] (3.5,10.5) -- (5,10.5);
\draw [short] (5,10.5) -- (5,9);
\draw [short] (5,9) -- (7.5,9);
\draw [->, >=Stealth] (7.5,9) -- (7.5,14.5);
\draw [->, >=Stealth] (9.75,9) -- (9.75,14.5);
\draw [short] (9.75,9) -- (11,9);
\draw [short] (11,9) -- (11,10.5);
\draw [short] (11,10.5) -- (12.5,10.5);
\draw [short] (12.5,10.5) -- (12.5,9);
\draw [short] (12.5,9) -- (13.75,9);
\draw [->, >=Stealth] (14,9) -- (14,14.5);
\draw [->, >=Stealth] (15.5,10.5) -- (15.5,14.5);
\draw [short] (15.5,10.5) -- (16,10.5);
\draw [short] (16,10.5) -- (16,9);
\draw [short] (16,9) -- (18.5,9);
\draw [short] (18.5,9) -- (18.5,10.5);
\draw [short] (18.5,10.5) -- (19,10.5);
\draw [->, >=Stealth] (19,10.5) -- (19,14.5);
\node [font=\normalsize] at (2,10.25) {V};
\node [font=\scriptsize] at (2.25,10.25) {0};
\node [font=\normalsize] at (5.5,13.5) {(i)};
\node [font=\normalsize] at (12,13.5) {(ii)};
\node [font=\normalsize] at (17,13.5) {(iii)};
\node [font=\normalsize] at (4.25,11) {L/3};
\node [font=\normalsize] at (10.25,11) {L/3};
\node [font=\normalsize] at (11.75,11) {L/3};
\node [font=\normalsize] at (15.75,11) {L/6};
\node [font=\normalsize] at (17.25,9.5) {2L/3};
\draw [<->, >=Stealth] (3.75,10.75) -- (4.75,10.75);
\draw [<->, >=Stealth] (9.75,10.75) -- (10.75,10.75);
\draw [<->, >=Stealth] (11,10.75) -- (12.5,10.75);
\draw [<->, >=Stealth] (15.5,10.75) -- (16,10.75);
\draw [<->, >=Stealth] (16.25,9.25) -- (18.25,9.25);
\node [font=\normalsize] at (2.25,9) {0};
\end{circuitikz}

}%
\end{figure}

\item In a survey work, three independent angles X, Y and Z were observed with weight $W_X$, $W_Y$, $W_Z$, respectively. The weight of the sum of angles X, Y and Z is given by:
\begin{enumerate}
    \item $\frac{1}{\brak{{\frac{1}{W_X}} + {\frac{1}{W_Y}} + {\frac{1}{W_Z}}}}$
    \item $\brak{\frac{1}{W_X} + \frac{1}{W_Y} + \frac{1}{W_Z}}$
    \item $W_X + W_Y + W_Z$
    \item $W_X^2 + W_Y^2 + W_Z^2$
\end{enumerate}
\item In a region with magnetic declination of $2\degree E$, the magnetic Fore bearing $\brak{FB}$ of a line AB was measured as $N79\degree 50'E$. There was local attraction at A. To determine the correct magnetic bearing of the line, a point O was selected at which there was no local attraction. The magnetic FN of the linw AO and OA were observed to be $S52\degree 40'E$ and $N50\degree 20'W$, respectively. What is the true FB of line AB?
\begin{enumerate}
    \item $N81\degree 50'E$
    \item $N82\degree 10'E$
    \item $N84\degree 10'E$
    \item $N77\degree 50'E$
\end{enumerate}


\end{enumerate}
\end{document}
