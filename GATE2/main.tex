%iffalse
\let\negmedspace\undefined
\let\negthickspace\undefined
\documentclass[journal,12pt,onecolumn]{IEEEtran}
\usepackage{cite}
\usepackage{amsmath,amssymb,amsfonts,amsthm}
\usepackage{algorithmic}
\usepackage{graphicx}
\usepackage{textcomp}
\usepackage{xcolor}
\usepackage{txfonts}
\usepackage{listings}
\usepackage{enumitem}
\usepackage{mathtools}
\usepackage{gensymb}
\usepackage{comment}
\usepackage[breaklinks=true]{hyperref}
\usepackage{tkz-euclide} 
\usepackage{listings}
\usepackage{gvv}       
\usepackage{circuitikz}
%\def\inputGnumericTable{}                                 
\usepackage[latin1]{inputenc}                                
\usepackage{color}                                            
\usepackage{array}                                            
\usepackage{longtable}                                       
\usepackage{calc}                                             
\usepackage{multirow}                                         
\usepackage{hhline}                                           
\usepackage{ifthen}                                           
\usepackage{lscape}
\usepackage{tabularx}
\usepackage{array}
\usepackage{float}


\newtheorem{theorem}{Theorem}[section]
\newtheorem{problem}{Problem}
\newtheorem{proposition}{Proposition}[section]
\newtheorem{lemma}{Lemma}[section]
\newtheorem{corollary}[theorem]{Corollary}
\newtheorem{example}{Example}[section]
\newtheorem{definition}[problem]{Definition}
\newcommand{\BEQA}{\begin{eqnarray}}
\newcommand{\EEQA}{\end{eqnarray}}
\newcommand{\define}{\stackrel{\triangle}{=}}
\theoremstyle{remark}
\newtheorem{rem}{Remark}

% Marks the beginning of the document
\begin{document}
\bibliographystyle{IEEEtran}
\vspace{3cm}

\title{2012-AE-27-39}
\author{AI24BTECH11011 - Himani Gourishetty}
\maketitle
\bigskip

\renewcommand{\thefigure}{\theenumi}
\renewcommand{\thetable}{\theenumi}
\begin{enumerate}
	\item The Lebesgue measure of the set $A=\cbrak{0\leq x\leq 1:x\sin\brak{\frac{\pi}{2x}}\geq 0}$ is
    \begin{enumerate}
        \item 0
        \item 1
        \item $\ln 2$
        \item $1-\ln\sqrt{2}$
    \end{enumerate}
    \item Which of the following statements are \textbf{TRUE}?\\
    P:The set $\cbrak{x \in \mathbb{R}:\abs{cox}\leq \frac{1}{2}}$ is a compact.\\
    Q: The set $\cbrak{x \in \mathbb{R}:\tan x \text{ is not differentiable}}$ is complete.\\
    R:The set $\cbrak{x \in \mathbb{R}:\sum_{n=0}^{\infty}\frac{\brak{-1}^nx^{2n+1}}{\brak{2n+1}!}\text{is convergent}} $ is bounded.\\
    S:The set $\cbrak{x \in \mathbb{R}:f\brak{x}=\cos x \text{has a local maxima}}$ is closed.
    \begin{enumerate}
        \item P and Q
        \item R and S
        \item Q and S
        \item P and S
    \end{enumerate}
    \item If a random variable $X$ assumes only positive integral values, with the probability $$P\brak{X=x}=\frac{2}{3}\brak{\frac{1}{3}}^{x-1},x=1,2,3,4,\cdots,$$ then $E\brak{X}$ is
    \begin{enumerate}
        \item $\frac{2}{9}$
        \item $\frac{2}{3}$
        \item 1
        \item $\frac{3}{2}$
    \end{enumerate}
    \item The probability density function of the random variable $X$ is
   \[ f(x)=
    \begin{cases}
        \frac{1}{\lambda}e^\frac{-x}{\lambda},x>0\\
        0, x\leq 0,
    \end{cases}
    \]
    where $\lambda>0$. For testing the hypothesis $H_0:\lambda=3,$ against $H_A:\lambda=5,$ a test is given as "Reject $H_0$ if $X\geq 4.5$". The probability of type I error and power of this text are, respectively,
    \begin{enumerate}
        \item 0.1353 and 0.4966
        \item 0.1827 and 0.379
        \item 0.2021 and 0.4493
        \item 0.2231 and 0.4066
    \end{enumerate}
    \item The order of the smallest possible non trivial group consisting elements $x$ and $y$ such that $x^7=y^2=e$ and $yx=x^4y$ is
    \begin{enumerate}
        \item 1
        \item 2
        \item 7
        \item 14
    \end{enumerate}
    \item The number of 5-Sylow subgroup(s) in a group of order 45 is 
    \begin{enumerate}
        \item 1
        \item 2
        \item 3
        \item 4
    \end{enumerate}
    \item The solution of the initial value problem 
    $$y''+2y^{\prime}+10y=6\delta\brak{t},\quad y\brak{0}=0,y^{\prime}\brak{0}=0,$$ where $\delta\brak{t}$ denotes the Dirac-delta function, is
    \begin{enumerate}
        \item $2e^{t}\sin 3t$
        \item $6e^{t}\sin 3t$
        \item $2e^{-t}\sin 3t$
        \item $6e^{-t}\sin 3t$
    \end{enumerate}
    \item Let $\omega=\cos\frac{2\pi}{3}+isin\frac{2\pi}{3},\vec{M}=\myvec{0 && i\\i && 0},\vec{N}=\myvec{\omega && 0\\ 0 &&\omega^{2}}$ and $G=\langle \vec{M} ,\vec{N} \rangle $ be the group generated by the matrix $\vec{M}$ and $\vec{N}$ under matrix multiplication. Then
    \begin{enumerate}
        \item $\frac{G}{Z}\brak{G}\cong C_6$
         \item $\frac{G}{Z}\brak{G}\cong S_3$
          \item $\frac{G}{Z}\brak{G}\cong C_2$
          \item $\frac{G}{Z}\brak{G}\cong C_4$ 
    \end{enumerate}
    \item The flux of the vector field $\vec{u}=x\hat{i}+y\hat{j}+z\hat{k}$ flowing out through the surface of the ellipsoid $$\frac{x^2}{a^2}+\frac{y^2}{b^2}+\frac{z^2}{c^2}=1,a>b>c>0,$$ is
    \begin{enumerate}
        \item $\pi abc$
        \item $2\pi abc$
        \item $3\pi abc$
        \item $4\pi abc$
    \end{enumerate}
    \item The integral surface satisfying the partial differential equation  $\frac{\partial z}{\partial x}+z^2\frac{\partial z}{\partial y}=0$ and passing through the straight line $x=1,y=z$ is
    \begin{enumerate}
        \item $\brak{x-1}z+z^2=y^2$
        \item $x^2+y^2-z^2=1$
        \item $\brak{y-z}x+x^2=1$
        \item $\brak{x-1}z^2+z=y$
    \end{enumerate}
\item The diffusion equation $$\frac{\partial^2 u}{\partial x^2}=\frac{\partial u}{\partial t}, u=u\brak{x,t},\quad u\brak{0,t}=0=u\brak{\pi,t}, \quad u\brak{x,0}=\cos x\sin 5x$$ admits the solution
\begin{enumerate}
    \item $\frac{e^{-36t}}{2}[\sin 6x+e^{20t}\sin 4x]$
    \item $\frac{e^{-36t}}{2}[\sin 4x+e^{20t}\sin 6x]$
    \item $\frac{e^{-20t}}{2}[\sin 3x+e^{15t}\sin 5x]$
    \item $\frac{e^{-36t}}{2}[\sin 5x+e^{20t}\sin x]$
\end{enumerate}
    \item Let $f\brak{x}$ and $xf\brak{x}$ be a particular solutions of a differential equation $$y''+R\brak{x}y^{\prime}+S\brak{x}y=0.$$ Then the solution of the differential equation $y''+R\brak{x}y^{\prime}+S\brak{x}y=f\brak{x}$ is
    \begin{enumerate}
        \item $y=\brak{\frac{-x^2}{2}+\alpha x+\beta}f\brak{x}$
        \item $y=\brak{\frac{x^2}{2}+\alpha x+\beta}f\brak{x}$
        \item $y=\brak{{-x^2}+\alpha x+\beta}f\brak{x}$
        \item $y=\brak{{x^2}+\alpha x+\beta}f\brak{x}$
    \end{enumerate}
    \item Let the Legendre equation $\brak{1-x^2}y"-2xy^{\prime}+n\brak{n+1}y=0$ have $n^{th}$ degree polynomial solution $y_n\brak{x}$ such that $y_n\brak{1}=3.$ If $\int_{-1}^{1}\brak{y^2_n\brak{x}+y^2_{n-1}\brak{x}}dx=\frac{144}{15}$, then n is
    \begin{enumerate}
        \item 1
        \item 2
        \item 3
        \item 4
    \end{enumerate}
\end{enumerate}

\end{document}
