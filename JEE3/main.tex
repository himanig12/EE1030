%iffalse
\let\negmedspace\undefined
\let\negthickspace\undefined
\documentclass[journal,12pt,onecolumn]{IEEEtran}
\usepackage{cite}
\usepackage{amsmath,amssymb,amsfonts,amsthm}
\usepackage{algorithmic}
\usepackage{graphicx}
\usepackage{textcomp}
\usepackage{xcolor}
\usepackage{txfonts}
\usepackage{listings}
\usepackage{enumitem}
\usepackage{mathtools}
\usepackage{gensymb}
\usepackage{comment}
\usepackage[breaklinks=true]{hyperref}
\usepackage{tkz-euclide} 
\usepackage{listings}
\usepackage{gvv}                                        
%\def\inputGnumericTable{}                                 
\usepackage[latin1]{inputenc}                                
\usepackage{color}                                            
\usepackage{array}                                            
\usepackage{longtable}                                       
\usepackage{calc}                                             
\usepackage{multirow}                                         
\usepackage{hhline}                                           
\usepackage{ifthen}                                           
\usepackage{lscape}
\usepackage{tabularx}
\usepackage{array}
\usepackage{float}


\newtheorem{theorem}{Theorem}[section]
\newtheorem{problem}{Problem}
\newtheorem{proposition}{Proposition}[section]
\newtheorem{lemma}{Lemma}[section]
\newtheorem{corollary}[theorem]{Corollary}
\newtheorem{example}{Example}[section]
\newtheorem{definition}[problem]{Definition}
\newcommand{\BEQA}{\begin{eqnarray}}
\newcommand{\EEQA}{\end{eqnarray}}
\newcommand{\define}{\stackrel{\triangle}{=}}
\theoremstyle{remark}
\newtheorem{rem}{Remark}

% Marks the beginning of the document
\begin{document}
\bibliographystyle{IEEEtran}
\vspace{3cm}

\title{02-24-2021-shift-2-1-15}
\author{AI24BTECH11011 - Himani Gourishetty}
\maketitle
\bigskip

\renewcommand{\thefigure}{\theenumi}
\renewcommand{\thetable}{\theenumi}
\begin{enumerate}
	\item Let $a,b \in \mathbb{R}$. If the mirror image of the point $\vec{P}\brak{a,6,9}$ with respect to the line $\frac{\brak{x-3}}{7}=\frac{\brak{y-2}}{5}=\frac{\brak{z-1}}{-9}$ is $\brak{20,b,-a,-9}$, then $\abs{a+b}$ is equal to:
    \begin{enumerate}
        \item 86
        \item 88
        \item 84
        \item 90
    \end{enumerate}
    \item Let f be a twice differentiable function defined on $\mathbb{R}$ such that $f\brak{0}=1,f^{\prime}\brak{0}=2$ and $f^{\prime}\brak{x}\neq0$ for all $x \in\mathbb{R}$. If $\abs{f\brak{x}f^{\prime}\brak{x}f^{\prime}\brak{x}f''\brak{x}}=0$, for all $x \in \mathbb{R}$, then the value of $f\brak{1}$ lies in the interval:
    \begin{enumerate}
        \item $\brak{9,12}$
        \item $\brak{6,9}$
        \item $\brak{3,6}$
        \item $\brak{0,3}$
    \end{enumerate}
    \item A possible value of $\tan\brak{\frac{1}{4}\sin^{-1}\frac{\sqrt{63}}{8}}$ is :
    \begin{enumerate}
        \item $\frac{1}{2\sqrt{2}}$
        \item $\frac{1}{\sqrt{7}}$
        \item $\sqrt{7}-1$
        \item $2\sqrt{2}-1$
    \end{enumerate}
    \item The probability that two randomly selected subsets of the set $\{1,2,3,4,5\}$ have exactly two elements in their intersection, is:
    \begin{enumerate}
        \item $\frac{65}{2^{7}}$
        \item $\frac{135}{2^{9}}$
        \item $\frac{65}{2^{8}}$
        \item $\frac{35}{2^{7}}$
    \end{enumerate}
    \item The vector equation of the plane passing through the intersection of the planes $\vec{r}.\brak{\hat{i}+\hat{j}+\hat{k}}=1$ and $\vec{r}\brak{\hat{i}-2\hat{j}}=-2$ and the points$\brak{1,0,2}$ is:
        \begin{enumerate}
        \item $\vec{r}.\brak{\hat{i}-7\hat{j}+3\hat{k}}=\frac{7}{3}$
        \item $\vec{r}.\brak{\hat{i}+7\hat{j}+3\hat{k}}=7$
        \item $\vec{r}.\brak{3\hat{i}+7\hat{j}+3\hat{k}}=7$
        \item $\vec{r}.\brak{\hat{i}+7\hat{j}+3\hat{k}}=\frac{7}{3}$
    \end{enumerate}
    \item If $\vec{P}$ is a point on the parabola $y = x^2 + 4$ which is closest to the straight line $y = 4x-1$, then the co-ordinates of $\vec{P}$ are :
    \begin{enumerate}
        \item $\brak{-2,8}$
        \item $\brak{1,5}$
        \item $\brak{3,13}$
        \item $\brak{2,8}$
    \end{enumerate}
    \item Let $a, b, c$ be in arithmetic progression. Let the centroid of the triangle with vertices $\brak{a, c}, \brak{2, b}$ and $\brak{a, b}$ be $\brak{\frac{10}{3}, \frac{7}{3}}$. If $\alpha,\beta$ are the roots of the equation $ax^2 + bx + 1 = 0$, then the value of $\alpha^2 + \beta^2 - \alpha\beta$ is:
    \begin{enumerate}
        \item $\frac{71}{256}$
        \item $\frac{-69}{256}$
        \item $\frac{69}{256}$
        \item $\frac{-71}{256}$ 
    \end{enumerate}
    \item The value of the integral, $\int_{1}^{3}\lfloor x^2 - 2x- 2\rfloor dx$ where $\lfloor x \rfloor$ denotes the greatest integer less than or equal to $x$, is :
    \begin{enumerate}
        \item -4
        \item -5
        \item $-\sqrt{2}-\sqrt{3}-1$
        \item $-\sqrt{2}-\sqrt{3}+1$
    \end{enumerate}
    \item Let $f:\mathbb{R}\rightarrow\mathbb{R}$ be defined as 
    \[
    f\brak{x}=
    \begin{cases}
        -55x,&\text{if } x<-5\\
        2x^3-3x^2-120x,&\text{if } -5\leq x\leq4\\
        2x^3-3x^2-36x-336,&\text{if } x>4
    \end{cases}
    \]
    Let $A=\{x \in \mathbb{R}:f$ is increasing\}.Then $A$ is equal to :
    \begin{enumerate}
        \item $\brak{-5,4} \cup \brak{4,\infty}$
        \item $\brak{-5,\infty}$
        \item $\brak{-\infty,-5} \cup \brak{4,\infty}$
        \item $\brak{-\infty,-5} \cup \brak{-4,\infty}$
    \end{enumerate}
    \item If the curve $y = ax^2 + bx + c,x \in \mathbb{R}$ passes through the point $\brak{1, 2}$ and the tangent line to this curve at origin is $y = x$, then the possible values of $a, b, c$ are:
    \begin{enumerate}
        \item $a = 1, b = 1, c = 0$
        \item $a = -1, b = 1, c = 1$
        \item $a = 1, b = 0, c = 1$
        \item $a = \frac{1}{2}, b = \frac{1}{2}, c = 1$
    \end{enumerate}
    \item  The negation of the statement $\neg p\land \brak{p\lor q}$ is:
    \begin{enumerate}
        \item $\neg p\land q$
        \item $p \land \neg q$
        \item $\neg p \lor q$
        \item $p \lor \neg q$
    \end{enumerate}
    \item For the system of linear equations:$x-2y = 1, x-y + kz = -2, ky + 4z = 6,k \in \mathbb{R}$\\
    Consider the following statements:
    \begin{enumerate}
    \item[(A)] The system has a unique solution if $k \neq 2, k \neq -2 $.
    \item[(B)] The system has a unique solution if $ k = -2 $.
    \item[(C)] The system has a unique solution if $k = 2 $.
    \item[(D)] The system has no solution if $ k = 2$.
    \item[(E)] The system has an infinite number of solutions if $ k \neq -2$.
    \end{enumerate}
    \begin{enumerate}
    \item (B) and (E) only
    \item (C) and (D) only
    \item (A) and (D) only
    \item (A) and (E) only
    \end{enumerate}
\item For which of the following curves, the line $x + \sqrt{3}y = 2\sqrt{3} $ is the tangent at the point $\brak{\frac{3\sqrt{3}}{2}, \frac{1}{2}}?$
\begin{enumerate}
    \item $x^2+9y^2=9$
    \item $2x^2-18y^2=9$
    \item $y^2=\frac{x}{6\sqrt{3}}$
    \item $x^2+y^2=7$
\end{enumerate}
\item The angle of elevation of a jet plane from a point $A$ on the ground is $60\degree$. After a flight of 20 seconds at the speed of $432 \frac{km}{hour}$, the angle of elevation changes to $30\degree$. If the jet plane is flying at a constant height, then its height is:
\begin{enumerate}
    \item $1200\sqrt{3}m$
    \item $1800\sqrt{3}m$
    \item $3600\sqrt{3}m$
    \item $2400\sqrt{3}m$
\end{enumerate}
\item For the statements p and q, consider the following compound statements:
\begin{enumerate}
    \item[(a)] $\brak{\neg q \land \brak{p \rightarrow q}} \rightarrow \neg p $
    \item[(b)] $\brak{\brak{p \lor q} \land \neg p}\rightarrow q$
\end{enumerate}
\begin{enumerate}
    \item $\brak{a}$ is a tautology but not $\brak{b}$
    \item $\brak{a}$ and $\brak{b}$ both are not tautologies
    \item $\brak{a}$ and $\brak{b}$ both are tautologies
    \item $\brak{a}$ is a tautology but not $\brak{b}$
\end{enumerate}
    \end{enumerate}

\end{document}
